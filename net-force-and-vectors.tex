\documentclass[12pt]{article}
\usepackage{graphicx,hyperref,amsmath,natbib,bm,url}
\usepackage[letterpaper,text={16cm,22.5cm},centering]{geometry}
\usepackage[compact,small]{titlesec}
\setlength{\parskip}{1.25ex}
\setlength{\parindent}{0em}
\clubpenalty = 10000
\widowpenalty = 10000
\usepackage[T1]{fontenc}
\usepackage[l2tabu,orthodox]{nag}  % force newer (and safer) LaTeX commands
\usepackage[utf8]{inputenc}        % set character set to support some UTF-8
                                   %   (unicode). Do NOT use this with
                                   %   XeTeX/LuaTeX!
\usepackage{babel}                 % multi-language support
\usepackage{sectsty}               % allow redefinition of section command formatting
\usepackage{tabularx}              % more table options
\usepackage{titling}               % allow redefinition of title formatting
\usepackage{imakeidx}              % create and index of words
\usepackage{amssymb}
\usepackage{xcolor}                % more colour options
\usepackage{enumitem}              % more list formatting options
\usepackage{tocloft}               % redefine table of contents, new list like objects

\begin{document}

\LARGE \begin{center}
Net Force and Vectors
\end{center}

\normalsize

Changes in \underline{~~~~~~~~~~~~~~~~} are produced by a \textbf{force} or combination of forces. A force, in the simplest sense, is a \textbf{push} or a \textbf{pull}. The source of a push or pull may be gravitational, electrical, magnetic, or simple muscular effort.

Here's a box of candy floating in free space. Let's suppose we exert a 5 N force on it (a bit more than 1 lb of force). \\

~~~~~~~~~~~~~~~~~ \large \textbf{\fbox{candy}}\normalsize $\rightarrow 5~N$ \\

A \underline{~~~~~~~~~~~~~~~~~~} is an arrow that shows the magnitude and direction of a measured property (such as force, velocity, acceleration, and much more). The magnitude is the \textbf{extent} of the property (how much it is). The 5 Newton force can produce a pick up in speed of the box. 

Suppose we exert a second identical force on the box. The pair of 5 Newton forces will \underline{~~~~~~~~~~~~~~~~~~~~~} the gain in speed.  \\

~~~~~~~~~~~~~~~~~ \large \textbf{\fbox{candy}}\normalsize $\rightrightarrows 10~N$ \\

The box reacts the same no matter if it is pushed by \textbf{two} 5 Newton forces or just \textbf{one} 10 Newton force. Both situations result in a net force of \underline{~~~~~~~} Newtons pushing on the box in the same direction. The forces simply add.

Suppose we pull on the box with two oppositely directed forces -- one 10 Newtons to the right and the other 5 Newtons to the left. \\

~~~~~~~~~~~~~~~~~ $5~N \leftarrow \large \textbf{\fbox{candy}}\normalsize \longrightarrow 10~N$ \\

How does the box move? Here, the forces \underline{~~~~~~~~~~~~~~~~~~~~~~~} and the box moves as if a single 5 Newton force acts on it. The net force on the box is 5 Newtons to the \underline{~~~~~~~~~~~~~~~~~}.

Suppose a pair of 5 Newton forces act in opposite directions on the box. \\

~~~~~~~~~~~~~~~~~ $5~N \leftarrow \large \textbf{\fbox{candy}}\normalsize \rightarrow 5~N$ \\

In this situation the forces cancel out and the net force on the box is \underline{~~~~~~~~~~~~~~} which means that the object experiences no change in \underline{~~~~~~~~~~~~~~~~~}. We say that the box is in mechanical \underline{~~~~~~~~~~~~~~~~~~~~~~~}.

Suppose that Nellie Newton hangs from a bar attached to a single rope. If Nellie's mass exerts a downward gravitational force of 300 $N$, then the force of tension on the rope is \underline{~~~~~~} $N$ in the upward direction.

Suppose three evenly spaced ropes support Nellie's mass so that the tension in each rope is the same. We know that the \textbf{sum} (addition) of the tensions have to add up to 300 Newtons. This means that each rope must hold \underline{~~~~~~~~~} $N$ of tension.

Lots of fun problems can be thought up by thinking about strings and string tension!


\end{document}