\documentclass[letterpaper, 12pt]{report}
\usepackage{enumitem}
\usepackage{multicol}
\usepackage{graphicx}
\usepackage{ulem} % allows for normal strikethroughs using \sout{}
\usepackage{cancel}
\usepackage{amsmath}
\usepackage{setspace}
\usepackage{geometry}
\geometry{
letterpaper,
left=25mm,
top=20mm,
bottom=25mm
}


\setstretch{1.15}

\begin{document}

\

\vspace{-0.3cm}

\begin{center}
\LARGE Pre-AP Chemistry Final Exam \\
\end{center}

\

\vspace{-0.5cm}

\begin{flushleft}
\normalsize

\textbf{1.} Name the three subatomic particles that comprise an atom. \\

\vspace{0.25cm}\textit{Solution.}\vspace{0.15cm} \\

Protons, neutrons, and electrons \\

\

\textbf{2.} Which subatomic particle carries a charge of $+1$? \\

\vspace{0.25cm}\textit{Solution.}\vspace{0.15cm} \\

Proton \\

\

\textbf{3.} Which subatomic particle carries a charge of $-1$? \\

\vspace{0.25cm}\textit{Solution.}\vspace{0.15cm} \\

Electron \\

\

\textbf{4.} Which two subatomic particles reside in the nucleus? \\

\vspace{0.25cm}\textit{Solution.}\vspace{0.15cm} \\

Proton and neutron \\

\

\textbf{5.} Briefly describe the nucleus. How much of the volume of an atom does the nucleus take up? How much of the mass of an atom does the nucleus hold? \\

\vspace{0.25cm}\textit{Solution.}\vspace{0.15cm} \\

The nucleus takes up a tiny fraction of the volume of an atom (less than 0.01\%) but contains more than 99.9\% of the mass of the atom. You don't have to remember the exact values, just the general idea. \\

\

\textbf{6.} How does the mass of a proton compare to the mass of a neutron? How does the mass of a proton compare to the mass of an electron? \\

\vspace{0.25cm}\textit{Solution.}\vspace{0.15cm} \\

The mass of a single proton and a single neutron are practically equivalent. Electrons are much lighter than protons and neutrons. One electron is $\sim$0.05\% the mass of a proton. Because of their tiny mass, electrons move at much faster speeds than protons.  \\

\

\textit{Detail:} Classical physics predicts that if an electron and a proton both carry the same momentum (mass $\times$ velocity), then  the electron would be traveling about 1800 times faster than the proton. This is not a very accurate approximation because classical physics has issues approximating particles moving at size-able fractions of the speed of light. \\

\

\textbf{7.} The periodic table is made up of boxes with symbols that describe atoms. If you read the periodic table like a book (from left-to-right and top-to-bottom) you will notice that one of these numbers goes up by 1 each time. What is the significance of this number? \\

\vspace{0.25cm}\textit{Solution.}\vspace{0.15cm} \\

The number which goes up by one each time is called the \textit{atomic number} and it tells you how many protons are in the nucleus of an atom. \\

\

\textit{Details:} The first element on the top left is Hydrogen. Hydrogen (H) has a nucleus with only 1 proton. The atom directly to the right of Hydrogen is Helium. Helium (He) has 2 protons in its nucleus. The next atom is Lithium (Li) which has 3 protons in its nucleus.  The number of protons in the nucleus of an atom is what gives an atom its name. The reason we tend to care more about the protons in the nucleus than the neutrons is because the charge on the protons is what gives atoms their chemical properties. \textit{Isotopes} are atoms with the same number of protons in the nucleus but different numbers of neutrons. Because neutrons are electrically neutral, they only effect the speed of reactivity (more neutrons means heavier atoms/molecules which means slightly slower reactivity). Different isotopes of the same atom tend to behave exactly the same in chemical reactions. It is for this reason that atoms get their names from the number of protons in their nucleus rather than the number of neutrons. \\

\

\textbf{8.} It is assumed that the atoms described on the periodic table are neutrally charged by default. If an atom is neutrally charged what does that mean about the number of protons and electrons in the atom? \\

\vspace{0.25cm}\textit{Solution.}\vspace{0.15cm} \\

If an atom or molecule is neutral it means that there is an equal number of protons and electrons. For example, if a lithium atom has 3 protons and 3 electrons then its charge is 0. \\

\

\textbf{9.} If an atom or molecule has a charge of +2 (which looks like $X^{2+}$) then how many more protons are there than electrons? \\

\vspace{0.25cm}\textit{Solution.}\vspace{0.15cm} \\

A charge of +2 means there are 2 more protons than electrons. For example, Be$^{2+}$ has 4 protons and 2 electrons.\\

\pagebreak

\textbf{10.} If an atom or molecule has a charge of -3 (which looks like $X^{3-}$) then how many more electrons are there than protons? \\

\vspace{0.25cm}\textit{Solution.}\vspace{0.15cm} \\

A charge of -3 means there are 3 more electrons than protons. For example, $N^{3-}$ has 7 protons and 10 electrons. \\

\


\textbf{11.} Heisenberg's uncertainty principle tells us that the act of measurement disrupts a system so that it is impossible to know the position and momentum of an electron at the same time. In other words, you can't know where an electron is and where it is going at the same time. What consequence does this have on modern scientific models which describe how electrons behave inside an atom? (Hint: when you can't know something for sure, the best way to model the information is using probability and statistics.)

\vspace{0.25cm}\textit{Solution.}\vspace{0.15cm} \\

Modern models of the atom represent the position of electrons as \textit{electron clouds}. Electron clouds are regions of space where electrons \textit{might} exist. The more shaded regions have a higher calculated probability of containing an electron. \\

\

\textit{Nerdy Details:} Scientists discovered how to calculate these probabilities by envisioning that particles propagates as waves. From this assumption, Schrodinger was able to define an equation which agrees with experimental data to a remarkable degree. Ever since, one of the major challenges of quantum physics and chemistry has been to understand \textit{why} the propagation of matter is better represented mathematically as waves. Some scientists believe that matter \textit{actually} behaves as waves and the only reason why humans perceive particles is because we exist at the ``object'' level of reality where lots of small interactions and different timelines/pathways collapse into the perception of single moments/states. Human beings live at a resolution where cause and effect is clear, but countless pathways exist which can take the universe between one state and the next. From the perspective of a human, the pathways always collapse into a state with identifiable characteristics and locations. However, we aren't able to tell which of the innumerable pathways were taken in order to arrive at the present state. Some really cool experiments have proven that, in fact, particles in the universe do not take ``definite trajectories" in space and time \textbf{unless} they are being measured! The only way our brains can build a coherent model of reality is to compress the branching pathways into some representation which performs statistics to determine probable outcomes. Our perception is constructed so that we can see the universal features which stay consistent across branching states. It find it fun to imagine the consequences of branching brains inside of a branching universe. \\

\pagebreak

\textbf{12.} What are electron energy levels? What does an electron's energy level tell you about its distance from the nucleus? How are electron energy levels organized on the periodic table? \\

\vspace{0.25cm}\textit{Solution.}\vspace{0.15cm} \\

According to quantum chemistry, electrons can only take on certain definite values of energy. Energy levels describe regions of space around the nucleus where electrons might be found. The more energy an electron has, the less bound it is by the nucleus. This means that higher energy levels describe regions of space which are farther away from the nucleus. As energy level increases, distance from the nucleus increases. The rows of the periodic table (called \textit{periods}) are arranged so that each new row introduces a new energy level. If you think about electron configuration, think about how each new element in the first column has electrons that follow the trend: $1s^{1}$ for H, $2s^{1}$ for Li, $3s^{1}$ for Na, etc. The leading number which goes up 1 each time is the \textit{energy level}. As you go down the periodic table the energy level increases which means the outer electrons get further and further from the nucleus. \\

\

\textbf{13.} What are \textit{valence electrons} and how are they different from \textit{core electrons}? \\

\vspace{0.25cm}\textit{Solution.}\vspace{0.15cm} \\

Valence electrons are the electrons in the outermost energy level. These are the electrons which are farthest from the nucleus and are free to interact between atoms. Core electrons are the electrons in the lower energy levels which are closer and more tightly bound to the nucleus. These electrons do not participate in bonding. \\

\

\textit{Extra Detail:} Core electrons reduce the pull that the nucleus has on valence electrons by screening/shielding those outside electrons from the nucleus's positive charge. This effect becomes more pronounced as atoms/molecules gain more and more core electrons. This, alongside the fact that electrostatic attraction decreases with distance, is why atoms towards the bottom of the periodic table require far less energy to eject electrons than atoms at the top of the periodic table. \\

\



\textbf{14.} What are bonds and what do they have to do with valence electrons? \\ 

\vspace{0.25cm}\textit{Solution.}\vspace{0.15cm} \\

Bonds are electrostatic forces which hold atoms together to form compounds and molecules. When atoms bond, there is either an exchange of valence electron(s) from one atom to another (called \textit{ionic bonding}) or sharing of valence electrons between atoms (called \textit{covalent bonding}). \\

\

\textit{Extra Details:} The reason that bonding occurs is because the overall potential energy \textit{after} bonding is lower. Nature always tries to minimize potential energy. The more protons a nucleus has, the more strongly it pulls on outside electrons. \textit{Electronegativity} is a description of how strongly a nucleus pulls on outside electrons. Electronegativity is determined by how many protons a nucleus has and how close an outside electron can get to that nucleus. \\

\

\textbf{15.} The periodic table is separated into 18 columns called \textit{groups}. The \textit{main group} elements are the atoms in groups 1-2 and 13-18. The periodic table is organized so that the groups correspond with a trend in valence electrons. What is the trend in valence electrons for the neutral main group elements? \\

\vspace{0.25cm}\textit{Solution.}\vspace{0.15cm}\\
\begin{center}
\begin{tabular}{|| c | c ||}
\hline
Group/Column Number & \# of Valence Electrons \\
\hline
1 & 1 \\
\hline
2 & 2 \\
\hline
13 & 3 \\
\hline
14 & 4 \\
\hline
15 & 5 \\
\hline
16 & 6 \\
\hline
17 & 7 \\
\hline
18 & 8 \\
\hline
\end{tabular}
\end{center}

\

\textbf{16.}  \textit{Ions} are formed when an atom/molecule gains or loses electrons so that the number of protons and electrons in the atom/molecule are not equal. The noble gases in group 18 have 8 valence electrons and are the most stable and nonreactive elements on the periodic table. The \textit{octet rule} tells us that atoms prefer to gain or lose electrons so that they form ions with 8 valence electrons. Draw a table that shows the common ions formed by atoms in Groups 1, 2, 13, 15, 16, and 17. \\

\vspace{0.25cm}\textit{Solution.}\vspace{0.15cm}
\begin{center}
\begin{tabular}{|| c | c | c | c ||}
\hline
Group/Column Number & Common Ion & \# of Valence Electrons & Change from Neutral Atom \\
\hline
1 & 1+ & 8 & Gives up 1 electron \\
\hline
2 & 2+ & 8 & Gives up 2 electrons \\
\hline
13 & 3+ & 8 & Gives up 3 electrons \\
\hline
15 & 3- & 8 & Obtains 3 extra electron\\
\hline
16 & 2- & 8 & Obtains 2 extra electrons\\
\hline
17 & 1- & 8 & Obtains 1 extra electrons \\
\hline
\end{tabular}
\end{center}

\textit{Note:} Particularly small atoms might end up with less than 8 valence electrons. For example $H^{+}$ has zero electrons, but this is still extremely stable! The general trend is still true. Notice how when metals give up electrons, they go from having 1, 2, or 3 valence electrons to having 8. This is because when they give up all of their valence electrons, their valence shells revert to the previous energy level. Since their previous energy level is completely full of electrons they revert to an octet! For example, when sodium (Na) gives up one electron it goes from having 1 valence electron ($3s^{1}$) to having the same electron configuration as Neon (which has 8 valence electrons). \\

\

\textbf{17.} What ionic compound is formed between Magnesium (Mg -- group 2) and Bromine (Br -- group 17)? \\

\vspace{0.25cm}\textit{Solution.}\vspace{0.15cm}\\

Magnesium is from group two and forms the ion $Mg^{2+}$. Bromine is from group 17 and forms the ion $Br^{-}$. Ionic compounds form so that the charge between the ions cancels out to zero. Since one Mg ion has a 2+ charge and one Br ion has a $1-$ charge, \textit{two} Br ions bond to one Mg ion. The resulting compound is $MgBr_{2}$ which has a neutral charge because $(+2) + (2)(-1) = +2 - 2 = 0$. \\

\

\textbf{18.} What ionic compound is formed between Aluminum (Al -- group 13) and Oxygen (O -- group 16)?

\vspace{0.25cm}\textit{Solution.}\vspace{0.15cm}\\

Aluminum is from group 13 and forms the ion $Al^{3+}$. Oxygen is from group 16 and forms the ion $O^{2-}$. Ionic compounds form so that the charge between the ions cancels out to zero. Since one Al ion has a 3+ charge and one oxygen ion has a $2-$ charge, \textit{two} Al ions bond to \textit{three} oxygen ions. The resulting compound is $Al_{2}O_{3}$ which has a neutral charge because $2(+3) + (3)(-2) = +6 - 6 = 0$. \\

\

\textbf{19.} A friend asks you to make her a cup of coffee using 3 teaspoons of sugar. You accidentally use 3 tablespoons instead. How many \textbf{extra} Calories does your friend consume when she accepts to drink it?

\vspace{0.15cm}
\fbox{1 tablespoon : 3 teaspoons} \hspace{0.25cm} \fbox{1 tablespoon of sugar : 49 Calories} \\

\vspace{0.35cm}\textit{Solution.}\vspace{0.15cm}\\

In order to determine how many more calories our friend drinks than what she requested we want to solve the expression: \\
\vspace{0.2cm}$(Calories~in~3~tbsp) - (Calories~in~3~tsp)$. \\

\

$Calories~in~3~tbsp = 3~\cancel{tbsp} \times \dfrac{49~Cal}{1~\cancel{tbsp}} = 147~Cal$ \\

\

We also need to know how many Calories are in 3 teaspoons. We could do the dimensional analysis for this part, but it is easy to see the answer. Since 3 teaspoons equals 1 tablespoon and 1 tablespoon of sugar is 49 Calories then it follows that 3 teaspoons of sugar is 49 Calories. \\

\

$147~Cal - 49~Cal = 98~Cal$ \\

\

Therefore, your mistake causes your friend to drink 98 unintended Calories. \\

\

\textbf{20.} The gram is most common unit for measuring mass in the chemistry lab. How does 1 gram relate to the mass of protons and neutrons? \\

\vspace{0.25cm}\textit{Solution.}\vspace{0.15cm}\\

Because protons and neutrons have equivalent mass, scientists have named this value the \textit{atomic mass unit} (amu). There are $6.022 \times 10^{23}$ atomic mass units in 1 gram. This is exactly the reason why Avogadro's Number is $6.022 \times 10^{23}$. \\

\

\textit{Details:} This conversion factor is what allows us to relate the masses that we measure in the lab to the masses of individual atoms and molecules. This is why the atomic masses on the periodic table are the same values whether you read them in amu or g/mol. It is important for chemists to be able to relate the mass due to protons and neutrons with the measured mass of a sample. If we know the substance the sample is made of, then we roughly know how many protons and neutrons there are in one molecule of that sample. Molar mass allows us to equate the atomic mass of that substance to an equal value with grams/mol as the unit. From there we can perform many convenient calculations. This is why moles and molar mass is important. \\

\

\textbf{21.} Calculate the molar mass of NH$_{3}$ and use it to determine the number of moles in 25 g of the substance. \\

\vspace{0.25cm}\textit{Solution.}\vspace{0.15cm}\\

\begin{center}
\begin{tabular}{|| c | c | c ||}
\hline
Atom & Atomic Mass & \# of Atoms in Molecule\\
\hline
N & 14.01 g/mol & 1 \\
H & 1.008 g/mol & 3 \\
\hline
\end{tabular}
\end{center}

\vspace{0.25cm}

$MM_{NH_{3}} = (1 \times 14.01) + (3 \times 1.008) = 17.025 g/mol$ \\

\vspace{0.35cm}

There are 17.025 grams in every 1 mol of NH$_{3}$. To find the number of moles in 25 grams, use the molar mass as a conversion factor (make sure that you set up your conversion so that units cancel correctly!). \\

\vspace{0.5cm}

$25~g \times \dfrac{1~mol}{17.025~g} = 1.5~mol~NH_{3}$
\pagebreak

\textbf{22.} Balance the following chemical reaction.\\

\

\underline{~~~~~} AgNO$_{3}$ (\textit{aq}) + \underline{~~~~~} H$_{2}$SO$_{4}$ (\textit{aq}) $\longrightarrow$ \underline{~~~~~} Ag$_{2}$SO$_{4}$ (\textit{s}) + \underline{~~~~~} HNO$_{3}$ (\textit{aq}) \\

\vspace{0.25cm}\textit{Solution.}\vspace{0.15cm}\\

First, we count the number of atoms on both sides of the skeleton equation. \\

\begin{center}
\begin{tabular}{|| c | c || c | c ||}
\hline
Reactant Side & Count & Product Side & Count \\
\hline
Ag & \cancel{1} 2 & Ag & 2 \\
NO$_{3}$ & \cancel{1} 2 & NO$_{3}$ & \cancel{1} 2 \\
H & 2 & H & \cancel{1} 2 \\
SO$_{4}$ & 1 & SO$_{4}$ & 1 \\
\hline
\end{tabular}

\ \vspace{0.35cm}

2 AgNO$_{3}$ (\textit{aq}) + H$_{2}$SO$_{4}$ (\textit{aq}) $\longrightarrow$ Ag$_{2}$SO$_{4}$ (\textit{s}) + 2 HNO$_{3}$ (\textit{aq}) \\
\end{center}

We need to balance the constituents on both sides of the reaction. We place a two in front of AgNO$_{3}$ on the reactant side to balance the Ag atom. This causes NO$_{3}$ to increase to two on the left hand side as well. This works out in our favor because when we compensate by putting a 2 in front of HNO$_{3}$ on the product side of the equation we end up balancing both the NO$_{3}$ ion and the H atom. Now we can see that the reaction is balanced. \\





\

\

\

\

\
\vspace{0.25cm}\textit{Solution.}\vspace{0.15cm}\\
\vspace{0.25cm}\textit{Solution.}\vspace{0.15cm}\\
\vspace{0.25cm}\textit{Solution.}\vspace{0.15cm}\\
\vspace{0.25cm}\textit{Solution.}\vspace{0.15cm}\\


\end{flushleft}

\end{document}









