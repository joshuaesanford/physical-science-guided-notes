\documentclass[12pt]{article}
\usepackage{graphicx,hyperref,amsmath,natbib,bm,url}
\usepackage[letterpaper,text={16cm,22.5cm},centering]{geometry}
\usepackage[compact,small]{titlesec}
\setlength{\parskip}{1.25ex}
\setlength{\parindent}{0em}
\clubpenalty = 10000
\widowpenalty = 10000
\usepackage[T1]{fontenc}
\usepackage[l2tabu,orthodox]{nag}  % force newer (and safer) LaTeX commands
\usepackage[utf8]{inputenc}        % set character set to support some UTF-8
                                   %   (unicode). Do NOT use this with
                                   %   XeTeX/LuaTeX!
\usepackage{babel}                 % multi-language support
\usepackage{sectsty}               % allow redefinition of section command formatting
\usepackage{tabularx}              % more table options
\usepackage{titling}               % allow redefinition of title formatting
\usepackage{imakeidx}              % create and index of words
\usepackage{xcolor}                % more colour options
\usepackage{enumitem}              % more list formatting options
\usepackage{tocloft}               % redefine table of contents, new list like objects

\begin{document}

\LARGE \begin{center}
Equilibrium Rule
\end{center}

\normalsize

\underline{~~~~~~~~~~~~~~~~~~~~~} is the force within a stretched rope.

Paul and Burl are standing on a scaffold painting a sign. Their positions, the scaffolding, and the ropes are all spaced out evenly. Both Paul and Burl twang their ropes like guitar strings. Burl's rope has a higher pitch than Paul's rope. This must mean that Burl weighs \underline{~~~~~~~~~~~~~~~~~~~} than Paul. 

Paul walks towards Burl to borrow one of his paint brushes. This means he moves his position and mass towards Burl and shifts the equilibrium of the scaffolding. This causes the tension in Burl's rope to become \underline{~~~~~~~~~~~~~~~~~~~} than it was before. 

The tension in the strings supports a system of objects with different masses. These objects include Paul, Burl, and the scaffolding. The strings will have the exact same tension \textbf{if and only if} the masses are positioned so that each string supports an equal amount of weight (assuming there are no other forces or disturbances in the system). In this scenario, when you pluck each string they will sound very similar because each string will produce the same frequency. 

If the masses become unevenly distributed then one string will end up holding more of the total tension than the other. This is how systems in equilibrium work. As the masses shift in position, the tension in the ropes automatically shifts so that all the upward and downward forces are evened out to 0. If the forces were not evened out to zero in the vertical direction then the scaffolding would either float upward or fall downward!

Scientists and mathematicians use the following notation to write the statement "the sum of all forces in the system are equal to zero": $\sum F = 0$ .

$F$ stands for "forces" and the symbol $\sum$ (the greek letter \textbf{sigma}) means "the sum of".

If Burl's rope increases in tension by 50 lbs then the tension in Paul's rope must \underline{~~~~~~~~~~~~~~~~} by 50 lbs. 

Remember, $F = ma$ where $F$ is force, $m$ is mass, and $a$ is acceleration. Acceleration due to gravity is -9.8 $m/s^{2}$. This means that the force of gravity on Burl, Paul, and the scaffolding is equal to the following.

$F_{Burl's~gravity} = m_{Burl} \times -9.8~m/s^{2}$

$F_{Paul's~gravity} = m_{Paul} \times -9.8~m/s^{2}$

$F_{scaffold's~gravity} = m_{scaffold} \times -9.8~m/s^{2}$

\small
$\sum F = F_{Burl} + F_{Paul} + F_{scaffold} + F_{rope~1} + F_{rope~2} = 0 $

Notice that the force of gravity on Burl, Paul, and the scaffolding are \textbf{constant}. However, the force on rope 1 and rope 2 can change as the position of the masses shift. Also note that these three gravitational forces will be \textbf{negative} because gravity pulls downward rather than upward. Rope tension is \textbf{positive} and in the upward direction which offsets the downward force of gravity and causes the sum of forces in the system to equal \textbf{zero}. This shows that the force of tension is preventing the scaffold from falling. \pagebreak

To recap, the equilibrium rule states that the sum of all forces that act on a system will balance to \underline{~~~~~~~~~~~~} if that system doesn't change its state of motion.

Burl, Paul, and the scaffold are part of a system that stays in equilibrium. Even more specifically, they are part of a system in equilibrium where all upward and downward forces cancel out to zero.

$\sum F = F_{upward} + F_{downward} = 0$

Burl, Paul, and the scaffold are the downward forces (which are negative). The upward force is the rope tension in the two ropes.

$F_{upward} = F_{string~tension~1} + F_{string~tension~2}$

$F_{downward} = F_{Burl} + F_{Paul} + F_{scaffold}$

If Burl is 140 lbs, then he exerts 140 lbs worth of gravitational force downwards. If Paul weighs 120 lbs, then he exerts 120 lbs worth of gravitational force downwards. If the scaffold weighs 60lbs, then it exerts 60 lbs worth of gravitational force downwards. 

This means that there is 320 lbs worth of gravitational force in the downward direction (the resulting force is a negative force because it is in the downward direction). If the total forces of the system are supposed to add up to zero it means that the string tension must exert 320 lbs worth of force in the upward (positive) direction. The resulting forces would all cancel out to zero.

$F_{upward} + F_{downward} = (F_{string~tension~1} + F_{string~tension~2}) + (F_{Burl} + F_{Paul} + F_{scaffold})$

$= (+320~lbs~worth~of~gravitational~force) + (-320~lbs~worth~of~gravitational~force) = 0$

Now for a problem. Remember that the total tension in the 2 ropes add up to 320 lbs worth of gravitational force.

$F_{Burl's~Rope} + F_{Paul's~Rope} = 320~lbs~worth~of~force$

Let's assume that the tension in Burl's rope holds 170 lbs worth of gravitational force. This means that Paul's rope must support \underline{~~~~~~~~~~~} lbs worth of gravitational force.

``Pounds worth of gravitational force" is a bit of a mouthful. I say it like this to keep the concept easy to understand. The normal scientific unit for a force is a Newton. This measurement uses kilograms instead of pounds for the unit of the weight. (Remember, $F = ma$.) The mass is multiplied by an acceleration which is in meters per second squared. This means that 1 Newton equals 1 kg $\times$ 1 meter/1 second/1 second. We can abbreviate newtons as $N$.

100 lbs of gravitational force converts to about 445 $N$.

Thanks for taking time to read along and add to the notes! Have a good rest of your day or evening!

\end{document}