\documentclass[10pt]{article}
\usepackage{graphicx,hyperref,amsmath,natbib,bm,url}
\usepackage[letterpaper,text={16cm,22.5cm},centering]{geometry}
\usepackage[compact,small]{titlesec}
\setlength{\parskip}{1.25ex}
\setlength{\parindent}{0em}
\clubpenalty = 10000
\widowpenalty = 10000
\usepackage[T1]{fontenc}
\usepackage[l2tabu,orthodox]{nag}  % force newer (and safer) LaTeX commands
\usepackage[utf8]{inputenc}        % set character set to support some UTF-8
                                   %   (unicode). Do NOT use this with
                                   %   XeTeX/LuaTeX!
\usepackage{babel}                 % multi-language support
\usepackage{sectsty}               % allow redefinition of section command formatting
\usepackage{tabularx}              % more table options
\usepackage{titling}               % allow redefinition of title formatting
\usepackage{imakeidx}              % create and index of words
\usepackage{amssymb}
\usepackage{xcolor}                % more colour options
\usepackage{enumitem}              % more list formatting options
\usepackage{tocloft}               % redefine table of contents, new list like objects

\begin{document}

\LARGE \begin{center}
Mass-Pulley Diagram
\end{center}
\normalsize Suppose there are two masses $m_{1}$ and $m_{2}$ which are equal to one another ($m_{1} = m_{2} = m$).

Mass $m_{1}$ sits on a table and is connected by rope and (frictionless) pulley to $m_{2}$ which is hanging off of the table. This simple diagram has been drawn for you on the board (and in the accompanying video if completing this assignment digitally). Draw this system in the free space on this paper and label the following force vectors.
\begin{enumerate}
\item Draw the \textbf{two} gravitational force vectors that point \textit{downwards} from each mass. These vectors should be equivalent because both masses are equivalent. Label these force vectors $mg$ (mass times gravitational acceleration where g is $9.8~m/s^{2}$).

\item Draw the \textbf{one} normal force vector that points \textit{upwards} from the mass which is supported by the table. This force is equal but opposite to the gravitational force because this block doesn't move up or down. Make sure that this vector is the same length as $mg$ but opposite in direction. Label this vector $N = mg$.

\item Draw the \textbf{two} string tension forces on \textit{opposing} sides of the frictionless pulley. Frictionless pulleys only change direction of a force. These two string forces only support the weight of one of the blocks ($m_{2}$). Therefore, each string force must be equal to $\frac{1}{2}mg$ (because $\frac{1}{2}mg + \frac{1}{2}mg = mg$). Make sure that the vectors are pointing the correct dirctions and are \textit{half} the length of the mg vectors. Remember, strings can only exert \textit{pulling}/\textit{supporting} forces. Label these vectors $T_{1} = \frac{1}{2}mg$ and $T_{2} = \frac{1}{2}mg$.

\item If the table is frictionless: (a) What is the \textit{magnitude} and \textit{direction} of the \textbf{net force} acting on $m_{1}$? (b) What is the \textit{magnitude} and \textit{direction} of the \textbf{net force} acting on $m_{2}$?

\item If the table is \textit{not} frictionless, a friction vector will act opposite to the string tension pulling on $m_{1}$. (a) If the force of friction is equal to $\frac{1}{2}mg$, will the masses move? (b) If the force of friction is less than $\frac{1}{2}mg$ will the masses move? 
\end{enumerate}






\end{document}