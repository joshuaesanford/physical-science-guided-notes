\documentclass[12pt]{article}
\usepackage{graphicx,hyperref,amsmath,natbib,bm,url}
\usepackage[letterpaper,text={16cm,22.5cm},centering]{geometry}
\usepackage[compact,small]{titlesec}
\setlength{\parskip}{1.25ex}
\setlength{\parindent}{0em}
\clubpenalty = 10000
\widowpenalty = 10000
\usepackage[T1]{fontenc}
\usepackage[l2tabu,orthodox]{nag}  % force newer (and safer) LaTeX commands
\usepackage[utf8]{inputenc}        % set character set to support some UTF-8
                                   %   (unicode). Do NOT use this with
                                   %   XeTeX/LuaTeX!
\usepackage{babel}                 % multi-language support
\usepackage{sectsty}               % allow redefinition of section command formatting
\usepackage{tabularx}              % more table options
\usepackage{titling}               % allow redefinition of title formatting
\usepackage{imakeidx}              % create and index of words
\usepackage{amssymb}
\usepackage{xcolor}                % more colour options
\usepackage{enumitem}              % more list formatting options
\usepackage{tocloft}               % redefine table of contents, new list like objects

\begin{document}

\LARGE \begin{center}
Newton's Laws of Motion
\end{center}

\normalsize

The tendency of an object to resist change in motion is called \underline{~~~~~~~~~~~~~~~~~~~~~~~}. 

Newton's first law of motion is known as the Law of \underline{~~~~~~~~~~~~~~~~~~~~~~~~}.

Newton's first law states that an object at rest stays at rest, or -- if already in motion -- continues to move at a constant speed in a straight line unless acted upon by a nonzero net force.

The key word is ``\textbf{continues}''. An object continues to do whatever its doing unless a force is exerted on it. A hockey puck at rest on the ice continues at rest if no force is exerted on it. If the hockey puck is struck by a force from a hockey stick it will accelerate while in contact with the force and then continue at a \underline{~~~~~~~~~~~~~~~~~~} velocity after it loses contact with the stick. If there is no friction, the hockey stick will continue at this velocity. If there \textit{is} friction then the friction will slow down the puck over time until the puck returns to rest.

\end{document}