\documentclass[12pt]{article}
\usepackage{graphicx,hyperref,amsmath,natbib,bm,url}
\usepackage[letterpaper,text={16cm,22.5cm},centering]{geometry}
\usepackage[compact,small]{titlesec}
\setlength{\parskip}{1.25ex}
\setlength{\parindent}{0em}
\clubpenalty = 10000
\widowpenalty = 10000
\usepackage[T1]{fontenc}
\usepackage[l2tabu,orthodox]{nag}  % force newer (and safer) LaTeX commands
\usepackage[utf8]{inputenc}        % set character set to support some UTF-8
                                   %   (unicode). Do NOT use this with
                                   %   XeTeX/LuaTeX!
\usepackage{babel}                 % multi-language support
\usepackage{sectsty}               % allow redefinition of section command formatting
\usepackage{tabularx}              % more table options
\usepackage{titling}               % allow redefinition of title formatting
\usepackage{imakeidx}              % create and index of words
\usepackage{amssymb}
\usepackage{xcolor}                % more colour options
\usepackage{enumitem}              % more list formatting options
\usepackage{tocloft}               % redefine table of contents, new list like objects

\begin{document}

\LARGE \begin{center}
Newton's Laws of Motion
\end{center}

\normalsize

The tendency of an object to resist change in motion is called \underline{~~~~~~~~~~~~~~~~~~~~~~~}. 

Newton's first law of motion is known as the Law of \underline{~~~~~~~~~~~~~~~~~~~~~~~~}.

Newton's first law states that an object at rest stays at rest. If the object is already in motion, then it continues to move at a constant speed in a straight line unless acted on by a nonzero net force.

The key word is ``\textbf{continues}''. An object continues to do whatever its doing unless a force is exerted on it. A hockey puck at rest on the ice will stay at rest if no force is exerted on it. The hockey puck will accelerate if struck by a force from a hockey stick. Once the puck is no longer in contact with the stick the force will no longer be acting on the puck. At this point, if there is no friction or obstacles, the hockey puck will continue at \underline{~~~~~~~~~~~~~~~~~~} velocity. If there \textit{is} friction then the friction will slow down the puck over time until the puck returns to rest.

Newton's first law tells us what kinds of motion require explanation. For example, ancient thinkers saw the motion of planets as natural. Newton noted that the planets move in curves, not straight lines. Therefore, Newton knew that the motion of the planets required a better explanation than ``natural heavenly motion." Some force must act on the planets in order to keep them going in curves rather than straight lines! This led Newton to the universal law of gravity. 

We see that many things around us change their speeds and directions. Their motion \textit{does} accelerate because they \textit{do} encounter forces. This brings us to Newton's second law, sometimes referred to as the Law of \underline{~~~~~~~~~~~~~~~~~~~~~~~~~~~~~~~}.

Newton's second law of motion states that the acceleration of an object is \underline{~~~~~~~~~~~~~~~~~~~~} proportional to the net force acting on the object. In other words, the greater the force, the greater the acceleration (and the lesser the force the lesser the acceleration). The direction of the net force also indicates the direction of the acceleration.

Newton's second law also states that the mass of an object is \underline{~~~~~~~~~~~~~~~~~~~~~~~~~} proportional to the acceleration of the object when met with a force. In other words, heavier objects will accelerate less than lighter objects when the same force is applied. The heavier an object, the less it will accelerate when a given force is applied. The lighter an object, the more it will accelerate when a given force is applied.

This can be shown in the following mathematical notation:

$a = \dfrac{F_{net}}{m}$, where $a$ is ``acceleration", $F_{net}$ is ``net force", and $m$ is ``mass" of the object.

Previously we said that a force is a push or a pull. Newton's second law gives a more precise definition and describes force based on the acceleration it can produce. \pagebreak



More often than not we view acceleration as being produced by a \underline{~~~~~~~~~~~~~~~~~~~~~~}. But could acceleration produce a force? 

If you swing a bat and hit a baseball the \underline{~~~~~~~~~~~~~~~~~~~~} of the impact produces the acceleration of the ball. In this case, force clearly produces an acceleration.

If you catch a ball, the ball's rapid \underline{~~~~~~~~~~~~~~~~~~~~~} exerts a force on the player's glove. Acceleration in this case clearly produces force. \textit{Which} produces \textit{which} can be arbitrary. 

This same idea applies to a car that crashes into a wall with strong foundation. If the car is moving at a constant velocity then it will have zero acceleration. However, this doesn't mean that the impact will result in zero force! The car will experience a rapid deceleration as a result of striking the mass of the wall. This deceleration will cause a resultant force on the car in the opposite direction it was moving. If the wall has strong enough foundation to withstand the force then the wall will not accelerate and therefore will not topple. The foundation must exert an equal and opposite force on the wall in order to protect it from toppling.

We often see Newton's second law expressed as $F = ma$. Some students interpret this to mean that force is a mass times an acceleration. This is not particularly helpful. It is more helpful to read this equation as saying ``the amount of force on a given mass relates to the acceleration produced by that force." For this reason, some people prefer the equation $a = \frac{F}{m}$ because it provides a cleaner intuition.

Newton's third law of motion states that whenever one object exerts a force on a second object, the second object exerts an \underline{~~~~~~~~~~~~~~~~} and \underline{~~~~~~~~~~~~~~~~~~~} force on the first. The common phrasing is ``to every action there is an equal and opposite reaction."

Newton's law tells us that forces always occur in \underline{~~~~~~~~~~~~}. Every force is part of an interaction between one thing and another. There is never only a single force in any situation.

If a heavy object collides with a light object, both objects experience the same force in opposite directions. The light object will accelerate in the direction of its force faster than the heavy object. Remember, heavier objects accelerate less than light objects when met with the same force. This can easily be checked in the equation. 

Think about how a cannon recoils when it fires a cannonball. The cannonball experiences a huge acceleration while the cannon accelerates to a much lesser degree in the opposite direction. This is because the cannon is a lot heavier than the cannonball!

It is a lot of fun to play with the idea of forces and collisions between objects such as two toy cars or billiard balls. Feel free to think up little experiments to try out on your own some time when you are bored!




\end{document}