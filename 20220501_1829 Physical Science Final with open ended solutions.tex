\documentclass[letterpaper, 12pt]{report}
\usepackage{enumitem}
\usepackage{multicol}
\usepackage{graphicx}
\usepackage{ulem} % allows for normal strikethroughs using \sout{}
\usepackage{cancel}
\usepackage{amsmath}
\usepackage{setspace}
\usepackage{geometry}
\geometry{
letterpaper,
left=25mm,
top=20mm,
bottom=25mm
}


\setstretch{1.5}

\begin{document}

\

\vspace{-1.75cm}

\begin{flushleft}
\large Name:~~~~~~~~~~~~~~~~~~~~~~~~~~~~~~~~~~~~~~~~~~~~~~~~~~~~~~~~~~~~~~~~~~~~~~~~~~~~~~~Period:
\end{flushleft}
\begin{center} \vspace{-0.5cm} \LARGE Physical Science Final Exam\end{center}

\

\vspace{-0.85cm}

\begin{flushleft}
\normalsize

\textbf{1.} Briefly describe what it means for a system of forces to be in equilibrium. \\

\vspace{0.3cm}

\textit{Solution.}\vspace{0.25cm} \\

When a system is in equilibrium it means that all of the forces in that system add up to 0. Because the total net force of the system is zero, the system is not accelerating or decelerating (speeding up or slowing down). \\

\vspace{0.65cm}
\hrule
\vspace{0.65cm}

\textbf{2.} Paul and Burl are standing on a scaffold painting a sign. The scaffold is held by two ropes on each side. Their positions, the scaffolding, and the ropes are all spaced out evenly. Both Paul and Burl twang their ropes like guitar strings. Burl's rope has a \textit{\textbf{higher}} pitch than Paul's rope. This must mean that Burl's weight is \underline{~~~~~~~~~~~~~~~~}. \\
\vspace{0.25cm}
(a) less than Paul's weight \\
(b) the same as Paul's weight \\
(c) more than Paul's weight \\
(d) the same as Paul's weight when Paul is standing on one foot \\

\vspace{0.65cm}
\hrule
\vspace{0.65cm}

\textbf{3.} The scaffolding that Burl and Paul are standing on is a system in equilibrium. If everything stays the same except that Burl's rope increases in tension by 50 lbs, then what must happen to the tension in Paul's rope? \\

\vspace{0.3cm}\textit{Solution.}\vspace{0.25cm} \\

Since the system is in equilibrium, all forces add up to 0. If Burl's rope increases in tension by 50 lbs then Paul's rope must \textit{decrease} by 50 lbs in order to maintain equilibrium. $+50 - 50 = 0$. \\

\vspace{0.65cm}
\hrule
\pagebreak

\textbf{4.} Paul walks towards Burl to borrow one of his paint brushes. This means he moves his position and mass towards Burl and shifts the equilibrium of the scaffolding. This causes the tension in Burl's rope to become \underline{~~~~~~~~~~~}. \\

(a) less than it was before\\
(b) the same as it was before \\
(c) more than it was before \\
(d) romantic \\

\vspace{0.65cm}
\hrule
\vspace{0.65cm}

\textbf{5.} Burl weighs 200 lbs, Paul weighs 150 lbs, and the scaffolding weighs 50 lbs. Assuming the mass is distributed evenly, then how many pounds of tension is their per rope? \\

\vspace{0.3cm}\textit{Solution.}\vspace{0.25cm} \\

Total lbs of force pulled down by gravity = $200 + 150 + 50 = 400~lbs$ \\

\vspace{0.2cm}

Number of ropes = 2

\vspace{0.2cm}

Tension per rope = $400~lbs \div 2 = \fbox{200~lbs}$ \\

\vspace{0.65cm}
\hrule
\vspace{0.65cm}

\textbf{6.} Perhaps the masses in the previous problem are \textbf{not} distributed evenly. If Burl's rope holds 220 lbs of tension, then how much tension is in Paul's rope? Remember, the system is still in equilibrium. \\

\vspace{0.3cm}\textit{Solution.}\vspace{0.25cm} \\

Total lbs of force pulled down by gravity = $400~lbs$ \\

\vspace{0.2cm}

Since all the forces have to cancel out (because the system is still in equilibrium), the tension in Paul and Burl's ropes must add up to 400 lbs. The 400 lbs of tension holds up the 400 lbs of weight being pulled down towards the earth. \\

\vspace{0.2cm}

$(Tension~in~Burl's~Rope) + (Tension~in~Paul's~Rope) = 400~lbs$ \\

\vspace{0.2cm}

$220~lbs + (Tension~in~Paul's~Rope) = 400~lbs$ \\
 
\vspace{0.2cm}

$Tension~in~Paul's~Rope = \fbox{180~lbs}$\\

\vspace{0.2817cm}
\hrule
\vspace{0.65cm}

\textbf{7.} The equation for Newton's Second Law of Motion is $F = ma$. Rearrange this equation to solve for mass and then rearrange the equation to solve for acceleration. \\

\vspace{0.3cm}\textit{Solution.}\vspace{0.25cm} \\

To solve for $m$, start with the given equation. \\

\vspace{0.25cm}

$F = ma$ \hspace{1.5cm} ; swap sides \\

\vspace{0.25cm}

$ma = F$ \hspace{1.5cm} ; multiply both sides of the equation by $\dfrac{1}{a}$ \\

\vspace{0.25cm}

$\dfrac{ma}{a} = \dfrac{F}{a}$ \hspace{1.35cm} ; $\dfrac{a}{a}$ cancels to 1 \\

\vspace{0.25cm}

$m = \dfrac{F}{a}$ \\

\

To solve for $a$, start with the given equation. \\

\vspace{0.25cm}

$F = ma$ \hspace{1.5cm} ; swap sides \\

\vspace{0.25cm}

$ma = F$ \hspace{1.5cm} ; multiply both sides of the equation by $\dfrac{1}{m}$ \\

\vspace{0.25cm}

$\dfrac{ma}{m} = \dfrac{F}{m}$ \hspace{1.35cm} ; $\dfrac{m}{m}$ cancels to 1 \\

\vspace{0.25cm}

$a = \dfrac{F}{m}$\\

\vspace{0.65cm}
\hrule
\vspace{0.65cm}

\textbf{8.} What acceleration is produced when an 8 N force is applied to an object with a mass of 4 kg? \\

\vspace{0.3cm}\textit{Solution.}\vspace{0.25cm} \\

$F = ma$ \hspace{1cm} ; Rearrange \\

\vspace{0.35cm}

$a = \dfrac{F}{m} = \dfrac{8~N}{4~kg} = \dfrac{8~kg \cdot m/s^{2}}{4~kg} = \hspace{0.25cm}$\fbox{$2~\dfrac{m}{s^{2}}$} \\

\vspace{0.65cm}
\hrule
\vspace{0.65cm}
\pagebreak

\textbf{9.} In the simplest sense, a \textit{force} is a \textbf{push} or a \textbf{pull}. Forces either \textit{cause} acceleration, or \textit{result} from deceleration. Give 3 examples of forces. \\

\vspace{0.3cm}\textit{Solution.}\vspace{0.25cm} \\

Examples: a baseball bat hitting a baseball, a foot pushing a pedal on a bicycle, the weight of gravity from a person's feet pushing against the ground, the positive end of a magnet pulling the negative end of another magnet, the force of friction between car tires and the road, a loaded spring being released, the tension on a rope holding up a sign, the force of air drag against an airplane, etc, etc. \\

\vspace{0.65cm}
\hrule
\vspace{0.65cm}

\textbf{10.} There are two forces acting on a 15 kg cinder block shown in the diagram below. One force pushes West at 5 N and the other force pushes East at 10 N. Assuming no friction, what is the magnitude and direction of the \textbf{acceleration} produced by the net total force on the cinder block? \\ 

\vspace{0.35cm}

$5~N \leftarrow \fbox{Cinder Block} \longrightarrow 10 N$ 

\vspace{0.3cm}\textit{Solution.}\vspace{0.25cm} \\

Because the forces are in opposite directions, the resulting net force is the difference between the two forces. 10 N East - 5 N West = 5 N East \\

\

Net Force: \hspace{0.3cm} $\fbox{Cinder Block} \rightarrow 5~N$ \\

\vspace{0.25cm}

$F = ma$ \\

\vspace{0.2cm}

$a = \dfrac{F}{m} = \dfrac{5~N}{15~kg} = \dfrac{5~kg \cdot m/s^{2}}{15~kg} = 0.33~\dfrac{m}{s^{2}}$ \\

\vspace{0.65cm}
\hrule
\vspace{0.65cm}

\pagebreak

\textbf{11.} Newton's First Law of Motion tells us that an object in motion will stay in motion in a \textit{straight line} unless acted on by some other force. However, astronomers observed that planets move in elliptical orbits. Which of the following is true?  \\

\

(a) Planets move through space unaffected by outside forces.\\
(b) Some outside force is responsible for the elliptical orbits.\\
(c) From applying Newton's First Law, Astronomers discovered that planets move in straight lines.\\
(d) Newton's First Law of Motion does not apply to celestial bodies. \\

\vspace{0.65cm}
\hrule
\vspace{0.65cm}

\textbf{12.} A force accelerates an object 10 $m/s^{2}$. How much would the same force accelerate an object with a mass twice as large? \\

\vspace{0.3cm}\textit{Solution.}\vspace{0.25cm} \\

Write the force equation and rearrange it for acceleration. The acceleration $a_{1}$ for the first object is given.\\

\vspace{0.25cm}

$F = ma$ \\
\vspace{0.3cm}
$a_{1} = \dfrac{F}{m} = 10~m/s^{2}$ \\
\vspace{0.5cm}

The acceleration $a_{2}$ for the second object has the same force but a mass ($2m$) that is twice as large. You can then pull out the $\frac{1}{2}$ in the equation which shows you that the answer for $a_{2}$ is to simply multiply the given acceleration by $\frac{1}{2}$. Therefore, you can see that, when force is constant, \textit{doubling} the \textbf{mass} \textit{halves} the \textbf{acceleration}. \\

\vspace{0.5cm}

$a_{2} = \dfrac{F}{2m} = \dfrac{1}{2} \left(\dfrac{F}{m}\right) = \dfrac{1}{2}a_{1} = \dfrac{1}{2}(10~m/s^{2}) = 5~m/s^{2}$ \\

\vspace{0.65cm}
\hrule
\vspace{0.65cm}

\pagebreak

\textbf{13.} You can calculate the \textit{force due to gravity} on an object using $F_{g} = mg$ (where $m$ is the mass of the object and $g$ is the value for gravitational acceleration). Take $g$ to equal $10~m/s^{2}$. What is the force of gravity on a person with a mass of 75 kg? \\

\vspace{0.3cm}\textit{Solution.}\vspace{0.25cm} \\

$F_{g} = mg$ \\
\vspace{0.2cm}
$F_{g} = 75~kg~ \times 10~\dfrac{m}{s^{2}}$\\
\vspace{0.2cm}
$F_{g} = 750~\dfrac{kg \cdot m}{s^{2}} = \fbox{750 N}$
\hspace{2cm} Remember, $\dfrac{kg \cdot m}{s^{2}}$ is equal to 1 Newton. \\

\vspace{0.65cm}
\hrule
\vspace{0.65cm}

\textbf{14.} The \textit{normal force} is called a \textit{support} force because it \textit{presses back} when some force is applied to a surface. Imagine you are standing on a scale inside of an elevator. The reading on the scale is the normal force. At equilibrium, the normal force is equal to the force of gravity. What happens to the reading on the scale when the elevator suddenly begins to accelerate \textit{\textbf{upwards}}? \\
\vspace{0.25cm}
(a) The reading on the scale increases because the upward acceleration of the elevator causes the scale to push harder against your feet.\\
(b) The reading on the scale decreases because the upward acceleration of the elevator takes pressure off of your feet.\\
(c) The reading on the scale stays the same because your body, the scale, and the elevator are traveling together.\\

\vspace{0.55cm}
\hrule
\vspace{0.55cm}

\textbf{15.} What happens to the reading on the scale when the elevator suddenly begins to accelerate \textit{\textbf{downwards}}? \\
\vspace{0.25cm}
(a) The reading on the scale increases because the downward acceleration of the elevator causes the scale to push harder against your feet.\\
(b) The reading on the scale decreases because the downward acceleration takes some pressure off of your feet.\\
(c) The reading on the scale stays the same because your body, the scale, and the elevator are traveling together.\\

\vspace{0.3655cm}
\hrule
\pagebreak

\textbf{16.} What is the difference between mass and weight? \\

\vspace{0.3cm}\textit{Solution.}\vspace{0.25cm} \\

Mass is the amount of matter in an object (the raw number of protons and neutrons). Weight is the force on an object due to gravity (which is determined by the mass of the planet/celestial body). For example, my \textit{mass} is about 75 kg, but my \textit{weight} due to gravity (on Earth) is about $75~kg \times 10~m/s^{2} = 750~N$. \\

\vspace{0.65cm}
\hrule
\vspace{0.65cm}

\textbf{17.} Imagine two identical boulders, one on earth and one floating freely in zero gravity space. Do the two boulders have equal mass or equal weight? \\

\vspace{0.3cm}\textit{Solution.}\vspace{0.25cm} \\

Weight changes depending on the strength of gravity. The boulder in space experiences no gravity and is therefore weightless. Mass stays the same regardless of the strength of gravity. Therefore the two boulders are equal in \textit{mass} but have different weights. \\

\vspace{0.65cm}
\hrule
\vspace{0.65cm}

\textbf{18.} A few skydivers dive from a high-flying, hovering helicopter. How is \textit{air resistance} affected as they fall faster and faster due to gravity? \\

\vspace{0.2cm}

(a) Air resistance increases. \\
(b) Air resistance decreases. \\
(c) Air resistance remains the same. \\

\vspace{0.65cm}
\hrule
\vspace{0.65cm}

\textbf{19.} As the skydivers fall faster and faster through the air, how does air resistance effect their acceleration? \\

\vspace{0.2cm}

(a) The change in air resistance causes the skydivers to accelerate at a faster rate (which causes them to fall faster). \\
(b) The change in air resistance causes the skydivers to decelerate until they reach a point of constant velocity with zero acceleration called \textit{terminal velocity}. \\
(c) There is no change in air resistance and the skydivers continue to accelerate at the rate of gravity. \\

\vspace{0.25555cm}
\hrule
\pagebreak


\textbf{20.} Fill in the blanks in the following statements using words from the word bank.

\vspace{0.35cm}

\begin{tabular}{|c c c|}
\hline
\textit{\textbf{energy}} & \textit{\textbf{work}} & \textit{\textbf{velocity}} \\
\textit{\textbf{distance}} & \textit{\textbf{motion}} & \textit{\textbf{acceleration}} \\
\textit{\textbf{Newton}} & \textit{\textbf{force}} & \textit{\textbf{Joule}}\\
\textit{\textbf{Kinetic Energy}} & \textit{\textbf{Potential Energy}} & \\
\hline
\end{tabular} 

\vspace{0.35cm}

\
The formula for \underline{\textit{\textbf{~velocity~}}} is $\dfrac{x_{f} - x_{i}}{t_{f} - t_{i}}$ which reads ``final distance minus initial distance divided by final time minus initial time" or ``the change in distance per unit time."
The formula for \underline{\textit{\textbf{~acceleration~}}} is $\dfrac{v_{f} - v_{i}}{t_{f} - t_{i}}$ which reads ``final velocity minus initial velocity divided by final time minus initial time" or ``the change in velocity per unit time." 
A \underline{\textit{\textbf{~force~}}} is a push or a pull which causes changes in the speed and direction of objects.
\underline{\textit{\textbf{~Energy~}}} is the capacity to do work. \underline{\textit{\textbf{~Work~}}} is done on an object when a force acts on it to produce a displacement (in other words, the force acts across a \underline{\textit{\textbf{~distance~}}}). If no \underline{\textit{\textbf{~motion~}}} occurs then the amount of work done is zero. The standard unit for force is called a \underline{\textit{\textbf{~Newton~}}} and is equal to $1~\dfrac{kg \cdot m}{s^{2}}$. This follows from $F = ma$ where mass is in kg and acceleration is in $m/s^{2}$. The standard unit for energy is called a \underline{\textit{\textbf{~Joule~}}} and is equal to $1~N \cdot m$ or $1~\dfrac{kg \cdot m^{2}}{s^{2}}$. This follows from $Work~Energy = Fd$ where $F$ is force and $d$ is \textit{displacement} (the distance the object has moved from some reference point). All forms of energy fall into two categories. \underline{\textit{\textbf{~Kinetic Energy~}}} refers to energy involved in motion. \underline{\textit{\textbf{~Potential Energy~}}} refers to energy that is stored up due to physical arrangement, attractions, and interactions. Work is a form of potential energy. Work, potential energy, and kinetic energy are all related. Work that ``could be done" is potential energy. Work ``being done" is the transferal of energy from potential to kinetic. \\

\vspace{0.65cm}
\pagebreak

\textbf{21.} Fill in the blanks in the following statements using words from the word bank.

\vspace{0.4cm}

\begin{tabular}{|c c c|}
\hline
\textit{\textbf{Kinetic Energy}} & \textit{\textbf{Potential Energy}} & \textit{\textbf{velocity}} \\
\textit{\textbf{minimum}} & \textit{\textbf{maximum}} & \\
\hline
\end{tabular} 

\vspace{0.5cm}

Suppose you climb inside of a tire and roll down a hill. At the top of the hill, before rolling down, the \underline{\textit{\textbf{~Potential Energy~}}} is at its maximum and \underline{\textit{\textbf{~Kinetic Energy~}}} is zero. In this circumstance, potential energy is the work that can be done due to gravity (given by $PE = mgh$). Gravitational Potential Energy is the multiplication of mass, gravitational acceleration, and height. By the time you roll halfway down the height of the hill, half of the potential energy will have been transferred into kinetic energy. As potential energy converts into kinetic energy there is an increase in \underline{\textit{\textbf{velocity}}}. At the bottom of the hill, potential energy will be at its \underline{\textit{\textbf{~minimum~}}} and kinetic energy will be at its \underline{\textit{\textbf{~maximum~}}}. \\

\vspace{0.65cm}

\textbf{22.} At the bottom of the hill, your velocity will be \underline{~~~~~~~~~~~~~~~~}. \\
\vspace{0.2cm}
(a) zero \\
(b) at its minimum \\
(c) speeding up \\
(d) at its maximum \\

\vspace{0.65cm}
\hrule
\vspace{0.65cm}

\textbf{23.} A bus travels 200 miles in 5 hours. Find the average velocity in miles per hour. \\

\vspace{0.3cm}\textit{Solution.}\vspace{0.25cm} \\

Velocity is the change in distance divided by the change in time. \\


\vspace{0.45cm}

$v = \dfrac{change~in~distance}{change~in~time} = \dfrac{200~mi}{5~hr} = 45~\dfrac{mi}{hr}$ 

\vspace{0.65cm}
\hrule
\pagebreak

\textbf{24.} An object with an initial speed of 10 $ft/s$ accelerates at 2 $ft/s^{2}$. What is its speed after 1 minute? \\

\vspace{0.3cm}\textit{Solution.}\vspace{0.25cm}

$1~min = 60~s$ \\
\vspace{0.3cm}
$Final~Speed = (Initial~Speed) + (Speed~Gained~Due~to~Acceleration)$\\
\vspace{0.3cm}
$Speed~Gained = acceleration \times time = \dfrac{2~ft}{s^{2}} \times 60~s = 120~\dfrac{ft}{s}$ \\
\vspace{0.3cm}
$Final~Speed = 10~\dfrac{ft}{s} + 120~\dfrac{ft}{s} = 130~\dfrac{ft}{s}$ \\

\vspace{0.65cm}
\hrule
\vspace{0.65cm}

\textbf{26.} What is the kinetic energy of a 100 kg  football player running 8 m/s? (This is roughly equal to a 220 lb wide receiver running 40 yds in roughly 4.6 sec.) $KE = \dfrac{1}{2}mv^{2}$. \\

\vspace{0.3cm}\textit{Solution.}\vspace{0.25cm}

$KE = \dfrac{1}{2}(100~kg)\left(\dfrac{8m}{s}\right)^{2} = 50~kg\left(\dfrac{64~m^{2}}{s^{2}}\right) = (50)(64)~\dfrac{kg \cdot m^{2}}{s^{2}} = \fbox{3200~J}$ \\

\vspace{0.65cm}
\hrule
\vspace{0.65cm}

\textbf{27.} Which of the following graphs shows runners moving at the same speed?
\includegraphics[scale=0.3575]{20220501_1744.png}

\hspace{2.25cm}(a) \hspace{3.55cm} (b) \hspace{3.65cm} (c) \hspace{3.525cm} (d) \\

\vspace{0.25cm}

\textbf{28.} A car is in motion. Select the response that describes its distance vs. time graph.
\begin{multicols}{2}
\includegraphics[scale=0.4]{20220501_1752.png}\\
\columnbreak
(a) The car is stopped.\\
(b) The car travels at a constant speed.\\
(c) The speed of the car is decreasing.\\
(d) The car is coming back.\\
\end{multicols}

\pagebreak

\textbf{29.} A car is in motion. Select the response that describes its distance vs. time graph. \\
\begin{center}
\includegraphics[scale=0.35]{20220501_1822.png} 
\end{center}
\vspace{-0.5cm}
(a) The car is stopped.\\
(b) The car travels at a constant speed.\\
(c) The speed of the car is decreasing.\\
(d) The car is coming back.\\

\vspace{0.65cm}

\textbf{30.} A car is in motion. Select the response that describes its \textit{\textbf{speed}} vs. time graph. \\
\begin{center}
\includegraphics[scale=0.375]{20220501_1826.png} 
\end{center}
\vspace{-0.5cm}
(a) The car is stopped. \\
(b) The car travels at a constant speed. \\
(c) The car is accelerating. \\
(d) The car is decelerating. 









\end{flushleft}

\end{document}